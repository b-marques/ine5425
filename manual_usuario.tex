\documentclass{article}
\usepackage[utf8]{inputenc}
\usepackage{indentfirst}
\usepackage{hyperref}
\usepackage{graphicx}
\usepackage[usenames, dvipsnames]{color}

\title{Manual de Usuário}
\author{Bruno Marques \\ \& \\ Gustavo Olegário}
\date{\today}

\renewcommand*\contentsname{Índice}

\begin{document}

    \begin{titlepage}
        \maketitle
        \thispagestyle{empty}
    \end{titlepage}
    
    \tableofcontents
    \clearpage
    \section{Sobre o programa}
        O programa de simualção desenvolvido em uma linguagem de propósito geral,
        tem como objetivo atender os requisitos solicitados pelo primeiro trabalho
        da disciplina INE5425 Modelagem e Simualção, do segundo semestre de 2017.
        O programa foi desenvolvido na linguagem C++ e é bem simples de utilizá-lo
        como podemos observar na figura abaixo:
        
        \begin{center}
          \includegraphics[scale=0.40]{foto.png}
        \end{center}
        
        O programa pode ser facilmente utilizado numa máquina que esteja rodando
        Windows. Basta que o usuário dê um duplo clique no executável que foi
        fornecido dentro do arquivo .zip.
    
    \section{Como utilizar o programa}
        Toda vez que o programa é inicializado, ele já vem os valores para cada
        preenchidos em cada campo e as respectivas curvas, isso foi desenvolvido
        desta forma para facilitar o teste do programa. Abaixo, há um detalhamento
        maior de cada campo e botões:
        
        \subsection{Sobre as curvas}
            Neste trabalho estão disponíveis 5 tipos de curvas para o usuário escolher o comportamento de uma variável, as quais são:
            \begin{itemize}
                \item Constante
                \item Normal
                \item Uniforme
                \item Triangular
                \item Exponencial
            \end{itemize}
            
            Sempre que o usuário escolher uma nova curva, a interface gráfica
            adaptará automaticamente o número de parâmetros. Utilizemos como 
            exemplo a curva triangular. Quando ela for selecionada, a interface
            ficará desta forma:
            
            \begin{center}
                \includegraphics[scale=0.70]{foto_triangular.png}
            \end{center}
            
        \subsection{Sobre os campos numéricos}
            Como já foi dito anteriormente, os campos de numéricos já são
            preenchidos automaticamente, mas o usuário pode alterar. O usuário
            pode incrementar ou decrementar unitarimanete valores clicando nas
            setas. Além disso, o usuário pode inserir valores diretamente utilizando
            o teclado. Independentemente da notação utilizada pelo usuário, o
            programa convertará pontos para vírgula.
            
        \subsection{Sobre a simulação}
            O usuário pode selecionar o atraso que acontecerá durante a simulação.
            Em outras palavras, qual será o atraso entre o processamento de dois
            eventos. Quanto maior esse valor, mais fácil o usuário poderá acompanhar
            os valores sendo atualizados. Além disso, existe uma barra de progresso,
            para que o usuário possa supervisionar o progresso da simulação:
            
            \begin{center}
                \includegraphics[scale=0.50]{foto_barra_de_carregamento.png}
            \end{center}
            
        \subsection{Sobre as variáveis de simulação}
            O usuário pode modificar as seguintes variáveis de simulação:
            \begin{itemize}
                \item Valor de tempo entre chegada das entidades do tipo 1
                \item Valor de tempo entre chegada das entidades do tipo 2
                \item Valor de tempo de serviço do servidor tipo 1 
                \item Valor de tempo de serviço do servidor tipo 2
                \item Valor de tempo entre falhas do servidor tipo 1
                \item Valor de tempo entre falhas do servidor tipo 2
                \item Valor de tempo de falha do servidor tipo 1
                \item Valor de tempo de falha do servidor tipo 2
            \end{itemize}
        
        \subsection{Sobre as variáveis de resultado}
            Na parte inferior da janela gráfica podemos avaliar todas as variáveis
            de resultado sendo atualizadas conforme o progresso da simulção:
            
            \begin{center}
                \includegraphics[scale=0.50]{foto_variaveis.png}
            \end{center}
        
\end{document}